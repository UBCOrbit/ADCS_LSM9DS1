\hypertarget{group__xTaskCreate}{}\section{x\+Task\+Create}
\label{group__xTaskCreate}\index{x\+Task\+Create@{x\+Task\+Create}}
task. h 
\begin{DoxyPre}
BaseType\_t xTaskCreate(
                          TaskFunction\_t pvTaskCode,
                          const char * const pcName,
                          uint16\_t usStackDepth,
                          void *pvParameters,
                          UBaseType\_t uxPriority,
                          TaskHandle\_t *pvCreatedTask
                      );\end{DoxyPre}


Create a new task and add it to the list of tasks that are ready to run.

Internally, within the Free\+R\+T\+OS implementation, tasks use two blocks of memory. The first block is used to hold the task\textquotesingle{}s data structures. The second block is used by the task as its stack. If a task is created using x\+Task\+Create() then both blocks of memory are automatically dynamically allocated inside the x\+Task\+Create() function. (see \href{http://www.freertos.org/a00111.html}{\tt http\+://www.\+freertos.\+org/a00111.\+html}). If a task is created using x\+Task\+Create\+Static() then the application writer must provide the required memory. x\+Task\+Create\+Static() therefore allows a task to be created without using any dynamic memory allocation.

See x\+Task\+Create\+Static() for a version that does not use any dynamic memory allocation.

x\+Task\+Create() can only be used to create a task that has unrestricted access to the entire microcontroller memory map. Systems that include M\+PU support can alternatively create an M\+PU constrained task using x\+Task\+Create\+Restricted().


\begin{DoxyParams}{Parameters}
{\em pv\+Task\+Code} & Pointer to the task entry function. Tasks must be implemented to never return (i.\+e. continuous loop).\\
\hline
{\em pc\+Name} & A descriptive name for the task. This is mainly used to facilitate debugging. Max length defined by config\+M\+A\+X\+\_\+\+T\+A\+S\+K\+\_\+\+N\+A\+M\+E\+\_\+\+L\+EN -\/ default is 16.\\
\hline
{\em us\+Stack\+Depth} & The size of the task stack specified as the number of variables the stack can hold -\/ not the number of bytes. For example, if the stack is 16 bits wide and us\+Stack\+Depth is defined as 100, 200 bytes will be allocated for stack storage.\\
\hline
{\em pv\+Parameters} & Pointer that will be used as the parameter for the task being created.\\
\hline
{\em ux\+Priority} & The priority at which the task should run. Systems that include M\+PU support can optionally create tasks in a privileged (system) mode by setting bit port\+P\+R\+I\+V\+I\+L\+E\+G\+E\+\_\+\+B\+IT of the priority parameter. For example, to create a privileged task at priority 2 the ux\+Priority parameter should be set to ( 2 $\vert$ port\+P\+R\+I\+V\+I\+L\+E\+G\+E\+\_\+\+B\+IT ).\\
\hline
{\em pv\+Created\+Task} & Used to pass back a handle by which the created task can be referenced.\\
\hline
\end{DoxyParams}
\begin{DoxyReturn}{Returns}
pd\+P\+A\+SS if the task was successfully created and added to a ready list, otherwise an error code defined in the file projdefs.\+h
\end{DoxyReturn}
Example usage\+: 
\begin{DoxyPre}
// Task to be created.
void vTaskCode( void * pvParameters )
\{
    for( ;; )
    \{
     // Task code goes here.
    \}
\}\end{DoxyPre}



\begin{DoxyPre}// Function that creates a task.
void vOtherFunction( void )
\{
static uint8\_t ucParameterToPass;
TaskHandle\_t xHandle = NULL;
\begin{DoxyVerb}// Create the task, storing the handle.  Note that the passed parameter ucParameterToPass
// must exist for the lifetime of the task, so in this case is declared static.  If it was just an
// an automatic stack variable it might no longer exist, or at least have been corrupted, by the time
// the new task attempts to access it.
xTaskCreate( vTaskCode, "NAME", STACK_SIZE, &ucParameterToPass, tskIDLE_PRIORITY, &xHandle );
configASSERT( xHandle );

// Use the handle to delete the task.
if( xHandle != NULL )
{
    vTaskDelete( xHandle );
}
\end{DoxyVerb}

\}
  \end{DoxyPre}
 