\hypertarget{group__xTaskAbortDelay}{}\section{x\+Task\+Abort\+Delay}
\label{group__xTaskAbortDelay}\index{x\+Task\+Abort\+Delay@{x\+Task\+Abort\+Delay}}
task. h 
\begin{DoxyPre}BaseType\_t xTaskAbortDelay( TaskHandle\_t xTask );\end{DoxyPre}


I\+N\+C\+L\+U\+D\+E\+\_\+x\+Task\+Abort\+Delay must be defined as 1 in \mbox{\hyperlink{FreeRTOSConfig_8h_source}{Free\+R\+T\+O\+S\+Config.\+h}} for this function to be available.

A task will enter the Blocked state when it is waiting for an event. The event it is waiting for can be a temporal event (waiting for a time), such as when v\+Task\+Delay() is called, or an event on an object, such as when x\+Queue\+Receive() or ul\+Task\+Notify\+Take() is called. If the handle of a task that is in the Blocked state is used in a call to x\+Task\+Abort\+Delay() then the task will leave the Blocked state, and return from whichever function call placed the task into the Blocked state.


\begin{DoxyParams}{Parameters}
{\em x\+Task} & The handle of the task to remove from the Blocked state.\\
\hline
\end{DoxyParams}
\begin{DoxyReturn}{Returns}
If the task referenced by x\+Task was not in the Blocked state then pd\+F\+A\+IL is returned. Otherwise pd\+P\+A\+SS is returned. 
\end{DoxyReturn}
