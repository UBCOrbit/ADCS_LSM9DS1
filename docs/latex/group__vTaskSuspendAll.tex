\hypertarget{group__vTaskSuspendAll}{}\section{v\+Task\+Suspend\+All}
\label{group__vTaskSuspendAll}\index{v\+Task\+Suspend\+All@{v\+Task\+Suspend\+All}}
task. h 
\begin{DoxyPre}void vTaskSuspendAll( void );\end{DoxyPre}


Suspends the scheduler without disabling interrupts. Context switches will not occur while the scheduler is suspended.

After calling v\+Task\+Suspend\+All () the calling task will continue to execute without risk of being swapped out until a call to x\+Task\+Resume\+All () has been made.

A\+PI functions that have the potential to cause a context switch (for example, v\+Task\+Delay\+Until(), x\+Queue\+Send(), etc.) must not be called while the scheduler is suspended.

Example usage\+: 
\begin{DoxyPre}
void vTask1( void * pvParameters )
\{
    for( ;; )
    \{
     // Task code goes here.\end{DoxyPre}



\begin{DoxyPre}     // ...\end{DoxyPre}



\begin{DoxyPre}     // At some point the task wants to perform a long operation during
     // which it does not want to get swapped out.  It cannot use
     // taskENTER\_CRITICAL ()/taskEXIT\_CRITICAL () as the length of the
     // operation may cause interrupts to be missed - including the
     // ticks.\end{DoxyPre}



\begin{DoxyPre}     // Prevent the real time kernel swapping out the task.
     vTaskSuspendAll ();\end{DoxyPre}



\begin{DoxyPre}     // Perform the operation here.  There is no need to use critical
     // sections as we have all the microcontroller processing time.
     // During this time interrupts will still operate and the kernel
     // tick count will be maintained.\end{DoxyPre}



\begin{DoxyPre}     // ...\end{DoxyPre}



\begin{DoxyPre}     // The operation is complete.  Restart the kernel.
     xTaskResumeAll ();
    \}
\}
  \end{DoxyPre}
 