\hypertarget{group__vTaskList}{}\section{v\+Task\+List}
\label{group__vTaskList}\index{v\+Task\+List@{v\+Task\+List}}
task. h 
\begin{DoxyPre}void vTaskList( char *pcWriteBuffer );\end{DoxyPre}


config\+U\+S\+E\+\_\+\+T\+R\+A\+C\+E\+\_\+\+F\+A\+C\+I\+L\+I\+TY and config\+U\+S\+E\+\_\+\+S\+T\+A\+T\+S\+\_\+\+F\+O\+R\+M\+A\+T\+T\+I\+N\+G\+\_\+\+F\+U\+N\+C\+T\+I\+O\+NS must both be defined as 1 for this function to be available. See the configuration section of the Free\+R\+T\+O\+S.\+org website for more information.

N\+O\+TE 1\+: This function will disable interrupts for its duration. It is not intended for normal application runtime use but as a debug aid.

Lists all the current tasks, along with their current state and stack usage high water mark.

Tasks are reported as blocked (\textquotesingle{}B\textquotesingle{}), ready (\textquotesingle{}R\textquotesingle{}), deleted (\textquotesingle{}D\textquotesingle{}) or suspended (\textquotesingle{}S\textquotesingle{}).

P\+L\+E\+A\+SE N\+O\+TE\+:

This function is provided for convenience only, and is used by many of the demo applications. Do not consider it to be part of the scheduler.

v\+Task\+List() calls ux\+Task\+Get\+System\+State(), then formats part of the ux\+Task\+Get\+System\+State() output into a human readable table that displays task names, states and stack usage.

v\+Task\+List() has a dependency on the sprintf() C library function that might bloat the code size, use a lot of stack, and provide different results on different platforms. An alternative, tiny, third party, and limited functionality implementation of sprintf() is provided in many of the Free\+R\+T\+O\+S/\+Demo sub-\/directories in a file called printf-\/stdarg.\+c (note printf-\/stdarg.\+c does not provide a full snprintf() implementation!).

It is recommended that production systems call ux\+Task\+Get\+System\+State() directly to get access to raw stats data, rather than indirectly through a call to v\+Task\+List().


\begin{DoxyParams}{Parameters}
{\em pc\+Write\+Buffer} & A buffer into which the above mentioned details will be written, in A\+S\+C\+II form. This buffer is assumed to be large enough to contain the generated report. Approximately 40 bytes per task should be sufficient. \\
\hline
\end{DoxyParams}
