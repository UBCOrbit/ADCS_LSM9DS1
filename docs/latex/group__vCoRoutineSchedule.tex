\hypertarget{group__vCoRoutineSchedule}{}\section{v\+Co\+Routine\+Schedule}
\label{group__vCoRoutineSchedule}\index{v\+Co\+Routine\+Schedule@{v\+Co\+Routine\+Schedule}}
croutine. h 
\begin{DoxyPre}
void vCoRoutineSchedule( void );\end{DoxyPre}


Run a co-\/routine.

v\+Co\+Routine\+Schedule() executes the highest priority co-\/routine that is able to run. The co-\/routine will execute until it either blocks, yields or is preempted by a task. Co-\/routines execute cooperatively so one co-\/routine cannot be preempted by another, but can be preempted by a task.

If an application comprises of both tasks and co-\/routines then v\+Co\+Routine\+Schedule should be called from the idle task (in an idle task hook).

Example usage\+: 
\begin{DoxyPre}
// This idle task hook will schedule a co-routine each time it is called.
// The rest of the idle task will execute between co-routine calls.
void vApplicationIdleHook( void )
\{
   vCoRoutineSchedule();
\}\end{DoxyPre}



\begin{DoxyPre}// Alternatively, if you do not require any other part of the idle task to
// execute, the idle task hook can call vCoRoutineScheduler() within an
// infinite loop.
void vApplicationIdleHook( void )
\{
   for( ;; )
   \{
       vCoRoutineSchedule();
   \}
\}
\end{DoxyPre}
 